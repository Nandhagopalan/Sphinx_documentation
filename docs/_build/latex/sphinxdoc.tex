%% Generated by Sphinx.
\def\sphinxdocclass{report}
\documentclass[letterpaper,10pt,english]{sphinxmanual}
\ifdefined\pdfpxdimen
   \let\sphinxpxdimen\pdfpxdimen\else\newdimen\sphinxpxdimen
\fi \sphinxpxdimen=.75bp\relax
\ifdefined\pdfimageresolution
    \pdfimageresolution= \numexpr \dimexpr1in\relax/\sphinxpxdimen\relax
\fi
%% let collapsible pdf bookmarks panel have high depth per default
\PassOptionsToPackage{bookmarksdepth=5}{hyperref}

\PassOptionsToPackage{warn}{textcomp}
\usepackage[utf8]{inputenc}
\ifdefined\DeclareUnicodeCharacter
% support both utf8 and utf8x syntaxes
  \ifdefined\DeclareUnicodeCharacterAsOptional
    \def\sphinxDUC#1{\DeclareUnicodeCharacter{"#1}}
  \else
    \let\sphinxDUC\DeclareUnicodeCharacter
  \fi
  \sphinxDUC{00A0}{\nobreakspace}
  \sphinxDUC{2500}{\sphinxunichar{2500}}
  \sphinxDUC{2502}{\sphinxunichar{2502}}
  \sphinxDUC{2514}{\sphinxunichar{2514}}
  \sphinxDUC{251C}{\sphinxunichar{251C}}
  \sphinxDUC{2572}{\textbackslash}
\fi
\usepackage{cmap}
\usepackage[T1]{fontenc}
\usepackage{amsmath,amssymb,amstext}
\usepackage{babel}



\usepackage{tgtermes}
\usepackage{tgheros}
\renewcommand{\ttdefault}{txtt}



\usepackage[Bjarne]{fncychap}
\usepackage{sphinx}

\fvset{fontsize=auto}
\usepackage{geometry}


% Include hyperref last.
\usepackage{hyperref}
% Fix anchor placement for figures with captions.
\usepackage{hypcap}% it must be loaded after hyperref.
% Set up styles of URL: it should be placed after hyperref.
\urlstyle{same}

\addto\captionsenglish{\renewcommand{\contentsname}{Contents:}}

\usepackage{sphinxmessages}
\setcounter{tocdepth}{0}



\title{Sphinx doc}
\date{May 28, 2022}
\release{v.1.0}
\author{Nandhagopalan Elangovan}
\newcommand{\sphinxlogo}{\vbox{}}
\renewcommand{\releasename}{Release}
\makeindex
\begin{document}

\pagestyle{empty}
\sphinxmaketitle
\pagestyle{plain}
\sphinxtableofcontents
\pagestyle{normal}
\phantomsection\label{\detokenize{index::doc}}


\sphinxstepscope


\chapter{Preprocess}
\label{\detokenize{notes/modules:preprocess}}\label{\detokenize{notes/modules::doc}}
\sphinxstepscope


\chapter{src}
\label{\detokenize{source/modules:src}}\label{\detokenize{source/modules::doc}}
\sphinxstepscope


\section{example module}
\label{\detokenize{source/example:module-example}}\label{\detokenize{source/example:example-module}}\label{\detokenize{source/example::doc}}\index{module@\spxentry{module}!example@\spxentry{example}}\index{example@\spxentry{example}!module@\spxentry{module}}
\sphinxAtStartPar
Example Google style docstrings.

\sphinxAtStartPar
This module demonstrates documentation as specified by the \sphinxhref{http://google.github.io/styleguide/pyguide.html}{Google Python
Style Guide}. Docstrings may extend over multiple lines. Sections are created
with a section header and a colon followed by a block of indented text.
\subsubsection*{Example}

\sphinxAtStartPar
Examples can be given using either the \sphinxcode{\sphinxupquote{Example}} or \sphinxcode{\sphinxupquote{Examples}}
sections. Sections support any reStructuredText formatting, including
literal blocks:

\begin{sphinxVerbatim}[commandchars=\\\{\}]
\PYGZdl{} python example\PYGZus{}google.py
\end{sphinxVerbatim}

\sphinxAtStartPar
Section breaks are created by resuming unindented text. Section breaks
are also implicitly created anytime a new section starts.
\index{module\_level\_variable1 (in module example)@\spxentry{module\_level\_variable1}\spxextra{in module example}}

\begin{fulllineitems}
\phantomsection\label{\detokenize{source/example:example.module_level_variable1}}
\pysigstartsignatures
\pysigline{\sphinxcode{\sphinxupquote{example.}}\sphinxbfcode{\sphinxupquote{module\_level\_variable1}}}
\pysigstopsignatures
\sphinxAtStartPar
Module level variables may be documented in
either the \sphinxcode{\sphinxupquote{Attributes}} section of the module docstring, or in an
inline docstring immediately following the variable.

\sphinxAtStartPar
Either form is acceptable, but the two should not be mixed. Choose
one convention to document module level variables and be consistent
with it.
\begin{quote}\begin{description}
\item[{Type}] \leavevmode
\sphinxAtStartPar
int

\end{description}\end{quote}

\end{fulllineitems}

\index{ExampleClass (class in example)@\spxentry{ExampleClass}\spxextra{class in example}}

\begin{fulllineitems}
\phantomsection\label{\detokenize{source/example:example.ExampleClass}}
\pysigstartsignatures
\pysiglinewithargsret{\sphinxbfcode{\sphinxupquote{class\DUrole{w}{  }}}\sphinxcode{\sphinxupquote{example.}}\sphinxbfcode{\sphinxupquote{ExampleClass}}}{\emph{\DUrole{n}{param1}}, \emph{\DUrole{n}{param2}}, \emph{\DUrole{n}{param3}}}{}
\pysigstopsignatures
\sphinxAtStartPar
Bases: \sphinxcode{\sphinxupquote{object}}

\sphinxAtStartPar
The summary line for a class docstring should fit on one line.

\sphinxAtStartPar
If the class has public attributes, they may be documented here
in an \sphinxcode{\sphinxupquote{Attributes}} section and follow the same formatting as a
function’s \sphinxcode{\sphinxupquote{Args}} section. Alternatively, attributes may be documented
inline with the attribute’s declaration (see \_\_init\_\_ method below).

\sphinxAtStartPar
Properties created with the \sphinxcode{\sphinxupquote{@property}} decorator should be documented
in the property’s getter method.
\index{attr1 (example.ExampleClass attribute)@\spxentry{attr1}\spxextra{example.ExampleClass attribute}}

\begin{fulllineitems}
\phantomsection\label{\detokenize{source/example:example.ExampleClass.attr1}}
\pysigstartsignatures
\pysigline{\sphinxbfcode{\sphinxupquote{attr1}}}
\pysigstopsignatures
\sphinxAtStartPar
Description of \sphinxtitleref{attr1}.
\begin{quote}\begin{description}
\item[{Type}] \leavevmode
\sphinxAtStartPar
str

\end{description}\end{quote}

\end{fulllineitems}

\index{attr2 (example.ExampleClass attribute)@\spxentry{attr2}\spxextra{example.ExampleClass attribute}}

\begin{fulllineitems}
\phantomsection\label{\detokenize{source/example:example.ExampleClass.attr2}}
\pysigstartsignatures
\pysigline{\sphinxbfcode{\sphinxupquote{attr2}}}
\pysigstopsignatures
\sphinxAtStartPar
Description of \sphinxtitleref{attr2}.
\begin{quote}\begin{description}
\item[{Type}] \leavevmode
\sphinxAtStartPar
\sphinxcode{\sphinxupquote{int}}, optional

\end{description}\end{quote}

\end{fulllineitems}


\sphinxAtStartPar
Example of docstring on the \_\_init\_\_ method.

\sphinxAtStartPar
The \_\_init\_\_ method may be documented in either the class level
docstring, or as a docstring on the \_\_init\_\_ method itself.

\sphinxAtStartPar
Either form is acceptable, but the two should not be mixed. Choose one
convention to document the \_\_init\_\_ method and be consistent with it.

\begin{sphinxadmonition}{note}{Note:}
\sphinxAtStartPar
Do not include the \sphinxtitleref{self} parameter in the \sphinxcode{\sphinxupquote{Args}} section.
\end{sphinxadmonition}
\begin{quote}\begin{description}
\item[{Parameters}] \leavevmode\begin{itemize}
\item {} 
\sphinxAtStartPar
\sphinxstyleliteralstrong{\sphinxupquote{param1}} (\sphinxstyleliteralemphasis{\sphinxupquote{str}}) \textendash{} Description of \sphinxtitleref{param1}.

\item {} 
\sphinxAtStartPar
\sphinxstyleliteralstrong{\sphinxupquote{param2}} (\sphinxcode{\sphinxupquote{int}}, optional) \textendash{} Description of \sphinxtitleref{param2}. Multiple
lines are supported.

\item {} 
\sphinxAtStartPar
\sphinxstyleliteralstrong{\sphinxupquote{param3}} (\sphinxcode{\sphinxupquote{list}} of \sphinxcode{\sphinxupquote{str}}) \textendash{} Description of \sphinxtitleref{param3}.

\end{itemize}

\end{description}\end{quote}
\index{attr3 (example.ExampleClass attribute)@\spxentry{attr3}\spxextra{example.ExampleClass attribute}}

\begin{fulllineitems}
\phantomsection\label{\detokenize{source/example:example.ExampleClass.attr3}}
\pysigstartsignatures
\pysigline{\sphinxbfcode{\sphinxupquote{attr3}}}
\pysigstopsignatures
\sphinxAtStartPar
Doc comment \sphinxstyleemphasis{inline} with attribute

\end{fulllineitems}

\index{attr4 (example.ExampleClass attribute)@\spxentry{attr4}\spxextra{example.ExampleClass attribute}}

\begin{fulllineitems}
\phantomsection\label{\detokenize{source/example:example.ExampleClass.attr4}}
\pysigstartsignatures
\pysigline{\sphinxbfcode{\sphinxupquote{attr4}}}
\pysigstopsignatures
\sphinxAtStartPar
Doc comment \sphinxstyleemphasis{before} attribute, with type specified
\begin{quote}\begin{description}
\item[{Type}] \leavevmode
\sphinxAtStartPar
list of str

\end{description}\end{quote}

\end{fulllineitems}

\index{attr5 (example.ExampleClass attribute)@\spxentry{attr5}\spxextra{example.ExampleClass attribute}}

\begin{fulllineitems}
\phantomsection\label{\detokenize{source/example:example.ExampleClass.attr5}}
\pysigstartsignatures
\pysigline{\sphinxbfcode{\sphinxupquote{attr5}}}
\pysigstopsignatures
\sphinxAtStartPar
Docstring \sphinxstyleemphasis{after} attribute, with type specified.
\begin{quote}\begin{description}
\item[{Type}] \leavevmode
\sphinxAtStartPar
str

\end{description}\end{quote}

\end{fulllineitems}

\index{example\_method() (example.ExampleClass method)@\spxentry{example\_method()}\spxextra{example.ExampleClass method}}

\begin{fulllineitems}
\phantomsection\label{\detokenize{source/example:example.ExampleClass.example_method}}
\pysigstartsignatures
\pysiglinewithargsret{\sphinxbfcode{\sphinxupquote{example\_method}}}{\emph{\DUrole{n}{param1}}, \emph{\DUrole{n}{param2}}}{}
\pysigstopsignatures
\sphinxAtStartPar
Class methods are similar to regular functions.

\begin{sphinxadmonition}{note}{Note:}
\sphinxAtStartPar
Do not include the \sphinxtitleref{self} parameter in the \sphinxcode{\sphinxupquote{Args}} section.
\end{sphinxadmonition}
\begin{quote}\begin{description}
\item[{Parameters}] \leavevmode\begin{itemize}
\item {} 
\sphinxAtStartPar
\sphinxstyleliteralstrong{\sphinxupquote{param1}} \textendash{} The first parameter.

\item {} 
\sphinxAtStartPar
\sphinxstyleliteralstrong{\sphinxupquote{param2}} \textendash{} The second parameter.

\end{itemize}

\item[{Returns}] \leavevmode
\sphinxAtStartPar
True if successful, False otherwise.

\end{description}\end{quote}

\end{fulllineitems}

\index{readonly\_property (example.ExampleClass property)@\spxentry{readonly\_property}\spxextra{example.ExampleClass property}}

\begin{fulllineitems}
\phantomsection\label{\detokenize{source/example:example.ExampleClass.readonly_property}}
\pysigstartsignatures
\pysigline{\sphinxbfcode{\sphinxupquote{property\DUrole{w}{  }}}\sphinxbfcode{\sphinxupquote{readonly\_property}}}
\pysigstopsignatures
\sphinxAtStartPar
Properties should be documented in their getter method.
\begin{quote}\begin{description}
\item[{Type}] \leavevmode
\sphinxAtStartPar
str

\end{description}\end{quote}

\end{fulllineitems}

\index{readwrite\_property (example.ExampleClass property)@\spxentry{readwrite\_property}\spxextra{example.ExampleClass property}}

\begin{fulllineitems}
\phantomsection\label{\detokenize{source/example:example.ExampleClass.readwrite_property}}
\pysigstartsignatures
\pysigline{\sphinxbfcode{\sphinxupquote{property\DUrole{w}{  }}}\sphinxbfcode{\sphinxupquote{readwrite\_property}}}
\pysigstopsignatures
\sphinxAtStartPar
Properties with both a getter and setter
should only be documented in their getter method.

\sphinxAtStartPar
If the setter method contains notable behavior, it should be
mentioned here.
\begin{quote}\begin{description}
\item[{Type}] \leavevmode
\sphinxAtStartPar
\sphinxcode{\sphinxupquote{list}} of \sphinxcode{\sphinxupquote{str}}

\end{description}\end{quote}

\end{fulllineitems}


\end{fulllineitems}

\index{ExampleError@\spxentry{ExampleError}}

\begin{fulllineitems}
\phantomsection\label{\detokenize{source/example:example.ExampleError}}
\pysigstartsignatures
\pysiglinewithargsret{\sphinxbfcode{\sphinxupquote{exception\DUrole{w}{  }}}\sphinxcode{\sphinxupquote{example.}}\sphinxbfcode{\sphinxupquote{ExampleError}}}{\emph{\DUrole{n}{msg}}, \emph{\DUrole{n}{code}}}{}
\pysigstopsignatures
\sphinxAtStartPar
Bases: \sphinxcode{\sphinxupquote{Exception}}

\sphinxAtStartPar
Exceptions are documented in the same way as classes.

\sphinxAtStartPar
The \_\_init\_\_ method may be documented in either the class level
docstring, or as a docstring on the \_\_init\_\_ method itself.

\sphinxAtStartPar
Either form is acceptable, but the two should not be mixed. Choose one
convention to document the \_\_init\_\_ method and be consistent with it.

\begin{sphinxadmonition}{note}{Note:}
\sphinxAtStartPar
Do not include the \sphinxtitleref{self} parameter in the \sphinxcode{\sphinxupquote{Args}} section.
\end{sphinxadmonition}
\begin{quote}\begin{description}
\item[{Parameters}] \leavevmode\begin{itemize}
\item {} 
\sphinxAtStartPar
\sphinxstyleliteralstrong{\sphinxupquote{msg}} (\sphinxstyleliteralemphasis{\sphinxupquote{str}}) \textendash{} Human readable string describing the exception.

\item {} 
\sphinxAtStartPar
\sphinxstyleliteralstrong{\sphinxupquote{code}} (\sphinxcode{\sphinxupquote{int}}, optional) \textendash{} Error code.

\end{itemize}

\end{description}\end{quote}
\index{msg (example.ExampleError attribute)@\spxentry{msg}\spxextra{example.ExampleError attribute}}

\begin{fulllineitems}
\phantomsection\label{\detokenize{source/example:example.ExampleError.msg}}
\pysigstartsignatures
\pysigline{\sphinxbfcode{\sphinxupquote{msg}}}
\pysigstopsignatures
\sphinxAtStartPar
Human readable string describing the exception.
\begin{quote}\begin{description}
\item[{Type}] \leavevmode
\sphinxAtStartPar
str

\end{description}\end{quote}

\end{fulllineitems}

\index{code (example.ExampleError attribute)@\spxentry{code}\spxextra{example.ExampleError attribute}}

\begin{fulllineitems}
\phantomsection\label{\detokenize{source/example:example.ExampleError.code}}
\pysigstartsignatures
\pysigline{\sphinxbfcode{\sphinxupquote{code}}}
\pysigstopsignatures
\sphinxAtStartPar
Exception error code.
\begin{quote}\begin{description}
\item[{Type}] \leavevmode
\sphinxAtStartPar
int

\end{description}\end{quote}

\end{fulllineitems}


\end{fulllineitems}

\index{example\_generator() (in module example)@\spxentry{example\_generator()}\spxextra{in module example}}

\begin{fulllineitems}
\phantomsection\label{\detokenize{source/example:example.example_generator}}
\pysigstartsignatures
\pysiglinewithargsret{\sphinxcode{\sphinxupquote{example.}}\sphinxbfcode{\sphinxupquote{example\_generator}}}{\emph{\DUrole{n}{n}}}{}
\pysigstopsignatures
\sphinxAtStartPar
Generators have a \sphinxcode{\sphinxupquote{Yields}} section instead of a \sphinxcode{\sphinxupquote{Returns}} section.
\begin{quote}\begin{description}
\item[{Parameters}] \leavevmode
\sphinxAtStartPar
\sphinxstyleliteralstrong{\sphinxupquote{n}} (\sphinxstyleliteralemphasis{\sphinxupquote{int}}) \textendash{} The upper limit of the range to generate, from 0 to \sphinxtitleref{n} \sphinxhyphen{} 1.

\item[{Yields}] \leavevmode
\sphinxAtStartPar
\sphinxstyleemphasis{int} \textendash{} The next number in the range of 0 to \sphinxtitleref{n} \sphinxhyphen{} 1.

\end{description}\end{quote}
\subsubsection*{Examples}

\sphinxAtStartPar
Examples should be written in doctest format, and should illustrate how
to use the function.

\begin{sphinxVerbatim}[commandchars=\\\{\}]
\PYG{g+gp}{\PYGZgt{}\PYGZgt{}\PYGZgt{} }\PYG{n+nb}{print}\PYG{p}{(}\PYG{p}{[}\PYG{n}{i} \PYG{k}{for} \PYG{n}{i} \PYG{o+ow}{in} \PYG{n}{example\PYGZus{}generator}\PYG{p}{(}\PYG{l+m+mi}{4}\PYG{p}{)}\PYG{p}{]}\PYG{p}{)}
\PYG{g+go}{[0, 1, 2, 3]}
\end{sphinxVerbatim}

\end{fulllineitems}

\index{function\_with\_pep484\_type\_annotations() (in module example)@\spxentry{function\_with\_pep484\_type\_annotations()}\spxextra{in module example}}

\begin{fulllineitems}
\phantomsection\label{\detokenize{source/example:example.function_with_pep484_type_annotations}}
\pysigstartsignatures
\pysiglinewithargsret{\sphinxcode{\sphinxupquote{example.}}\sphinxbfcode{\sphinxupquote{function\_with\_pep484\_type\_annotations}}}{\emph{\DUrole{n}{param1}\DUrole{p}{:}\DUrole{w}{  }\DUrole{n}{int}}, \emph{\DUrole{n}{param2}\DUrole{p}{:}\DUrole{w}{  }\DUrole{n}{str}}}{{ $\rightarrow$ bool}}
\pysigstopsignatures
\sphinxAtStartPar
Example function with PEP 484 type annotations.
\begin{quote}\begin{description}
\item[{Parameters}] \leavevmode\begin{itemize}
\item {} 
\sphinxAtStartPar
\sphinxstyleliteralstrong{\sphinxupquote{param1}} \textendash{} The first parameter.

\item {} 
\sphinxAtStartPar
\sphinxstyleliteralstrong{\sphinxupquote{param2}} \textendash{} The second parameter.

\end{itemize}

\item[{Returns}] \leavevmode
\sphinxAtStartPar
The return value. True for success, False otherwise.

\end{description}\end{quote}

\end{fulllineitems}

\index{function\_with\_types\_in\_docstring() (in module example)@\spxentry{function\_with\_types\_in\_docstring()}\spxextra{in module example}}

\begin{fulllineitems}
\phantomsection\label{\detokenize{source/example:example.function_with_types_in_docstring}}
\pysigstartsignatures
\pysiglinewithargsret{\sphinxcode{\sphinxupquote{example.}}\sphinxbfcode{\sphinxupquote{function\_with\_types\_in\_docstring}}}{\emph{\DUrole{n}{param1}}, \emph{\DUrole{n}{param2}}}{}
\pysigstopsignatures
\sphinxAtStartPar
Example function with types documented in the docstring.

\sphinxAtStartPar
\sphinxhref{https://www.python.org/dev/peps/pep-0484/}{PEP 484} type annotations are supported. If attribute, parameter, and
return types are annotated according to \sphinxhref{https://www.python.org/dev/peps/pep-0484/}{PEP 484}, they do not need to be
included in the docstring:
\begin{quote}\begin{description}
\item[{Parameters}] \leavevmode\begin{itemize}
\item {} 
\sphinxAtStartPar
\sphinxstyleliteralstrong{\sphinxupquote{param1}} (\sphinxstyleliteralemphasis{\sphinxupquote{int}}) \textendash{} The first parameter.

\item {} 
\sphinxAtStartPar
\sphinxstyleliteralstrong{\sphinxupquote{param2}} (\sphinxstyleliteralemphasis{\sphinxupquote{str}}) \textendash{} The second parameter.

\end{itemize}

\item[{Returns}] \leavevmode
\sphinxAtStartPar
The return value. True for success, False otherwise.

\item[{Return type}] \leavevmode
\sphinxAtStartPar
bool

\end{description}\end{quote}

\end{fulllineitems}

\index{module\_level\_function() (in module example)@\spxentry{module\_level\_function()}\spxextra{in module example}}

\begin{fulllineitems}
\phantomsection\label{\detokenize{source/example:example.module_level_function}}
\pysigstartsignatures
\pysiglinewithargsret{\sphinxcode{\sphinxupquote{example.}}\sphinxbfcode{\sphinxupquote{module\_level\_function}}}{\emph{\DUrole{n}{param1}}, \emph{\DUrole{n}{param2}\DUrole{o}{=}\DUrole{default_value}{None}}, \emph{\DUrole{o}{*}\DUrole{n}{args}}, \emph{\DUrole{o}{**}\DUrole{n}{kwargs}}}{}
\pysigstopsignatures
\sphinxAtStartPar
This is an example of a module level function.

\sphinxAtStartPar
Function parameters should be documented in the \sphinxcode{\sphinxupquote{Args}} section. The name
of each parameter is required. The type and description of each parameter
is optional, but should be included if not obvious.

\sphinxAtStartPar
If *args or **kwargs are accepted,
they should be listed as \sphinxcode{\sphinxupquote{*args}} and \sphinxcode{\sphinxupquote{**kwargs}}.

\sphinxAtStartPar
The format for a parameter is:

\begin{sphinxVerbatim}[commandchars=\\\{\}]
\PYG{n}{name} \PYG{p}{(}\PYG{n+nb}{type}\PYG{p}{)}\PYG{p}{:} \PYG{n}{description}
    \PYG{n}{The} \PYG{n}{description} \PYG{n}{may} \PYG{n}{span} \PYG{n}{multiple} \PYG{n}{lines}\PYG{o}{.} \PYG{n}{Following}
    \PYG{n}{lines} \PYG{n}{should} \PYG{n}{be} \PYG{n}{indented}\PYG{o}{.} \PYG{n}{The} \PYG{l+s+s2}{\PYGZdq{}}\PYG{l+s+s2}{(type)}\PYG{l+s+s2}{\PYGZdq{}} \PYG{o+ow}{is} \PYG{n}{optional}\PYG{o}{.}

    \PYG{n}{Multiple} \PYG{n}{paragraphs} \PYG{n}{are} \PYG{n}{supported} \PYG{o+ow}{in} \PYG{n}{parameter}
    \PYG{n}{descriptions}\PYG{o}{.}
\end{sphinxVerbatim}
\begin{quote}\begin{description}
\item[{Parameters}] \leavevmode\begin{itemize}
\item {} 
\sphinxAtStartPar
\sphinxstyleliteralstrong{\sphinxupquote{param1}} (\sphinxstyleliteralemphasis{\sphinxupquote{int}}) \textendash{} The first parameter.

\item {} 
\sphinxAtStartPar
\sphinxstyleliteralstrong{\sphinxupquote{param2}} (\sphinxcode{\sphinxupquote{str}}, optional) \textendash{} The second parameter. Defaults to None.
Second line of description should be indented.

\item {} 
\sphinxAtStartPar
\sphinxstyleliteralstrong{\sphinxupquote{*args}} \textendash{} Variable length argument list.

\item {} 
\sphinxAtStartPar
\sphinxstyleliteralstrong{\sphinxupquote{**kwargs}} \textendash{} Arbitrary keyword arguments.

\end{itemize}

\item[{Returns}] \leavevmode
\sphinxAtStartPar

\sphinxAtStartPar
True if successful, False otherwise.

\sphinxAtStartPar
The return type is optional and may be specified at the beginning of
the \sphinxcode{\sphinxupquote{Returns}} section followed by a colon.

\sphinxAtStartPar
The \sphinxcode{\sphinxupquote{Returns}} section may span multiple lines and paragraphs.
Following lines should be indented to match the first line.

\sphinxAtStartPar
The \sphinxcode{\sphinxupquote{Returns}} section supports any reStructuredText formatting,
including literal blocks:

\begin{sphinxVerbatim}[commandchars=\\\{\}]
\PYG{p}{\PYGZob{}}
    \PYG{l+s+s1}{\PYGZsq{}}\PYG{l+s+s1}{param1}\PYG{l+s+s1}{\PYGZsq{}}\PYG{p}{:} \PYG{n}{param1}\PYG{p}{,}
    \PYG{l+s+s1}{\PYGZsq{}}\PYG{l+s+s1}{param2}\PYG{l+s+s1}{\PYGZsq{}}\PYG{p}{:} \PYG{n}{param2}
\PYG{p}{\PYGZcb{}}
\end{sphinxVerbatim}


\item[{Return type}] \leavevmode
\sphinxAtStartPar
bool

\item[{Raises}] \leavevmode\begin{itemize}
\item {} 
\sphinxAtStartPar
\sphinxstyleliteralstrong{\sphinxupquote{AttributeError}} \textendash{} The \sphinxcode{\sphinxupquote{Raises}} section is a list of all exceptions
    that are relevant to the interface.

\item {} 
\sphinxAtStartPar
\sphinxstyleliteralstrong{\sphinxupquote{ValueError}} \textendash{} If \sphinxtitleref{param2} is equal to \sphinxtitleref{param1}.

\end{itemize}

\end{description}\end{quote}

\end{fulllineitems}

\index{module\_level\_variable2 (in module example)@\spxentry{module\_level\_variable2}\spxextra{in module example}}

\begin{fulllineitems}
\phantomsection\label{\detokenize{source/example:example.module_level_variable2}}
\pysigstartsignatures
\pysigline{\sphinxcode{\sphinxupquote{example.}}\sphinxbfcode{\sphinxupquote{module\_level\_variable2}}\sphinxbfcode{\sphinxupquote{\DUrole{w}{  }\DUrole{p}{=}\DUrole{w}{  }98765}}}
\pysigstopsignatures
\sphinxAtStartPar
Module level variable documented inline.

\sphinxAtStartPar
The docstring may span multiple lines. The type may optionally be specified
on the first line, separated by a colon.
\begin{quote}\begin{description}
\item[{Type}] \leavevmode
\sphinxAtStartPar
int

\end{description}\end{quote}

\end{fulllineitems}



\chapter{Indices and tables}
\label{\detokenize{index:indices-and-tables}}\begin{itemize}
\item {} 
\sphinxAtStartPar
\DUrole{xref,std,std-ref}{genindex}

\item {} 
\sphinxAtStartPar
\DUrole{xref,std,std-ref}{modindex}

\item {} 
\sphinxAtStartPar
\DUrole{xref,std,std-ref}{search}

\end{itemize}


\renewcommand{\indexname}{Python Module Index}
\begin{sphinxtheindex}
\let\bigletter\sphinxstyleindexlettergroup
\bigletter{e}
\item\relax\sphinxstyleindexentry{example}\sphinxstyleindexpageref{source/example:\detokenize{module-example}}
\end{sphinxtheindex}

\renewcommand{\indexname}{Index}
\printindex
\end{document}